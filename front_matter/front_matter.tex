% Title page
\maketitle

% Intentionally blank page
\blankpagecount

% Examination Page
\examinationpage

% Copyrights page
\copyrightspage

% After here everything is left-aligned
\RaggedRight


%%%%%%%%%%%%%%%%%%%%%%%%%%%%%%%%%%%%%%%%%%%%%%%%%%%%%%%%%%%%%%%%%
%% Περίληψη
%%%%%%%%%%%%%%%%%%%%%%%%%%%%%%%%%%%%%%%%%%%%%%%%%%%%%%%%%%%%%%%%%
\chapter*{Περίληψη}
\addcontentsline{toc}{chapter}{Περίληψη}

Ο σκοπός της διπλωµατικής εργασίας ήταν η ανάπτυξη µεθοδολογίας για την
ανίχνευση, αναγνώριση και καταγραφή σηµάτων σε δεδοµένο φάσµα συχνοτήτων. Η
µεθοδολογία αυτή εφαρµόστηκε για την εύρεση παρεµβολών στο φάσµα συχνοτήτων του
κυψελωτού συστήµατος κινητών επικοινωνιών DCS 1800 στην ευρύτερη περιοχή του
Λεκανοπεδίου Αττικής. Για το σκοπό αυτό πραγµατοποιήθηκαν εξωτερικές µετρήσεις σε
επιλεγµένα σηµεία. Η επεξεργασία των µετρήσεων κατέδειξε την ύπαρξη παρεµβολών στο
φάσµα του DCS 1800.

Συγκεκριµένα, έγινε µελέτη του κυψελωτού συστήµατος DCS 1800, παρουσιάστηκαν
τα είδη παρεµβολών, παράχθηκαν σήµατα DCS 1800 στο εργαστήριο µε χρήση ψηφιακής
γεννήτριας και έγινε εργαστηριακός έλεγχος – εξαγωγή χαρακτηριστικών καµπυλών
παθητικών (BF φίλτρο, οµοαξονικά καλώδια) και ενεργών στοιχείων (LNA) µε χρήση HP
Network Analyzer. Επίσης, καταγράφηκαν οι παρεµβολές σε PC µέσω HP Spectrum
Analyzer και κατάλληλου λογισµικού.

Η µεθοδολογία αυτή µπορεί να γίνει οδηγός για την πραγµατοποίηση εξωτερικών ή
εσωτερικών µετρήσεων σε οποιοδήποτε φάσµα συχνοτήτων, µε απλές αλλαγές στις
ρυθµίσεις του αναλυτή φάσµατος. Η γενικότητα της µεθοδολογίας έγκειται στο γεγονός ότι
έχουν καταγραφεί όλα τα βήµατα, από την προστασία του προσωπικού και του εξοπλισµού
µέχρι αναλυτικά όλα τα στάδια διεξαγωγής των µετρήσεων.


\vspace{20ex}
\section*{Λέξεις Κλειδιά}
Μετρήσεις, Παρεµβολές, DCS 1800, Κινητή Τηλεφωνία, Ηλεκτροµαγνητική Ακτινοβολία,
Ηλεκτροµαγνητικό Υπόβαθρο, Κινητές Επικοινωνίες, Επεξεργασία Μετρήσεων


\clearpage

\blankpagecount

%%%%%%%%%%%%%%%%%%%%%%%%%%%%%%%%%%%%%%%%%%%%%%%%%%%%%%%%%%%%%%%%%
%% Abstract
%%%%%%%%%%%%%%%%%%%%%%%%%%%%%%%%%%%%%%%%%%%%%%%%%%%%%%%%%%%%%%%%%
\chapter*{\en{Abstract}}
\addcontentsline{toc}{chapter}{\en{Abstract}}

\begin{otherlanguage}{english}
    The scope of this thesis was the development of a methodology in order
    to detect, recognize and record signals in a certain spectrum. This
    methodology was applied to the finding of interferences into the spectrum of
    the cellular mobile communications system DCS 1800 in the wider area of
    the Attika basin. For that purpose, outdoor measurements were carried out at
    selected sites. The processing of the measurements showed the existence of
    interferences into the DCS 1800 spectrum.

    Specifically, the DCS 1800 cellular system was studied and the
    interference theory was presented. Furthermore, DCS 1800 signals were
    generated at the laboratory with the use of a digital generator and a
    laboratory test - extraction of the characteristic curves - of passive
    (Bandpass Filter, co-axial cables) and active elements (LNA) was carried
    out, using a HP Network Analyzer. Moreover, the interferences were
    recorded on a hard disk through a HP Spectrum Analyzer and proper
    software.

    This methodology can be used as a guide for carrying out both outdoor and
    indoor measurements in any spectrum, by making simple changes at the
    function keys of the Spectrum Analyzer. The usefulness of the methodology
    is the specification of the procedure of the measurements in steps, from the
    protection of the personnel and the equipment up to the analytical stages of
    the measurements procedure.

    \vspace{20ex}
    \section*{\en{Keywords}}
    Measurements, Interference, DCS 1800, Mobile Communications,
    Electromagetic Radiation, Measurements processing
\end{otherlanguage}
\clearpage

\blankpagecount

%%%%%%%%%%%%%%%%%%%%%%%%%%%%%%%%%%%%%%%%%%%%%%%%%%%%%%%%%%%%%%%%%
%% Ευχαριστίες
%%%%%%%%%%%%%%%%%%%%%%%%%%%%%%%%%%%%%%%%%%%%%%%%%%%%%%%%%%%%%%%%%
\chapter*{Ευχαριστίες}
\addcontentsline{toc}{chapter}{Ευχαριστίες}

Ευχαριστώ την οικογένεια μου και τους καθηγητές μου.
\clearpage

\blankpagecount


%%%%%%%%%%%%%%%%%%%%%%%%%%%%%%%%%%%%%%%%%%%%%%%%%%%%%%%%%%%%%%%%%
%% TABLE OF CONTENTS (TOC), LISTS OF FIGURES, TABLES, ETC.
%%%%%%%%%%%%%%%%%%%%%%%%%%%%%%%%%%%%%%%%%%%%%%%%%%%%%%%%%%%%%%%%%

\tableofcontents

\listoffigures

\listoftables

\clearpage

